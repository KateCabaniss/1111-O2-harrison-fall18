\documentclass[12pt]{article}
\usepackage[margin=0.75in]{geometry}
\usepackage{graphicx}
\usepackage{float}
\setlength{\parindent}{0mm}

\begin{document}

{\centering
\large Physics I: Lab 07 (Extra Credit (Optional)) \par
\large Momentum \par
}
\hfill \break \vspace{-4mm}

Suppose you are standing on a frictionless sled ($M_{you+sled} = 100$ kg) moving at 0.5 m/s.
You also have 22 blocks of mass 12 kg (i.e. $M_{total} = M_{you+sled}$ + 22(12 kg)).
You can boost your velocity by throwing the blocks horizontally at velocity 11 m/s in the opposite direction of your motion.
Create the following two plots: velocity vs the number of blocks thrown, and momentum vs the number of blocks thrown.
\hfill \break

Run the following simulation to test your results:
\begin{itemize}
\item Make sure you have Java JDK 1.8 installed
\item Download rocket-app.zip from https://github.com/naharrison/discrete-rocket/releases
\item Unzip the file and open rocket.jar
\item Click the green button to throw a block
\end{itemize}

\end{document}
